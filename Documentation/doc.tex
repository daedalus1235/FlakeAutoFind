\documentclass{article}
\usepackage{hw}

\title{FlakeAutoFind}
\author{Charles Yang}
\date{Last updates: \today}

\begin{document}
\maketitle
\setcounter{tocdepth}{1}
\tableofcontents

\section{General Algorithm}
The image is flattened using an infered background level. After flattening, morphological transformations are applied. Opening reduces the amount of small, white pixels, eroding reduces the size of bigger clumps of white pixels, and closing makes the flakes more solid and bold. 

\section{Flattening}
The symmetry of the background is assumed to that of an ellipse. Similar to a circle, where the radius is defined
\[s^2=(x-x_c)^2+(y-y_c)^2\]
we define a distance coordinate
\[s^2=(x-x_c)^2+r^2(y-y_c)^2\]
where \(r\) is the aspect ratio of the ellipse, given
\[r = \frac{\text{width}}{\text{height}}\]

In this way, we consider the background to be a function of the elliptical distance. To maintain smoothness, the background is modeled as the sum of even terms:
\[B_N(s)=\sum_{n=0}^N a_ns^{2n}\]

Clearly, the first \(n\) coefficients can be solved for using a system of linear equations generated by \(n\) samples of the background. This yields the equation
\[B = SA\]
where \(B\) is a vector of measured background values, \(A\) is a vector of the coefficients, and \(S\) is the Vandermonde matrix of the values \(s_i^2\). 



Unfortunately, only considering the lowest order (\(s^2\)) leads to inaccuracy around the borders of the image. 
\subsection{Sampling by area}
If we take samples in using the sequence
\[s^2, 2s^2, 3s^2,\dots, ns^2\]
we can represent the equations as
\begin{equation}
	[S\vert I]=\left[ \begin{matrix}1 & 1 & 1 & \cdots \\ 1 & 2 & 4 & \cdots\\	1 & 3 & 9 & \cdots\\ \vdots & \vdots & \vdots & \ddots \end{matrix}\,\middle\vert\,\begin{matrix}1 & 0 & 0 & \cdots \\ 0 & 1 & 0 & \cdots \\ 0 & 0 & 1 & \cdots \\ \vdots & \vdots & \vdots & \ddots\end{matrix}\right]
\end{equation}
Using row reduction, we can make the vandermonde matrix into an upper triangular matrix, and the coefficient matrix has a very simple form:
\begin{equation}
	[T\vert C]=\left[\begin{matrix} 1 & 1 & 1 & 1 & 1 & \cdots \\
			0 & 1 & 3 & 7 & 15 &\cdots \\
			0 & 0 & 2 & 12 & 50 &\cdots \\
			0 & 0 & 0 & 6 & 30 & \cdots \\
			0 & 0 & 0 & 0 & 144 & \cdots \\
		\vdots & \vdots & \vdots & \vdots & \vdots & \ddots \end{matrix}
		\,\,\middle\vert\,\,\begin{matrix}
			1 & 0 & 0 & 0 & 0 & \cdots \\
			-1 & 1 & 0 & 0 & 0 & \cdots \\
			1 & -2 & 1 & 0 & 0 & \cdots \\
			-1 & 3 & -3 & 1 & 0 & \cdots \\
			1 & -4 & 6 & -4 & 1 & \cdots \\
			\vdots & \vdots & \vdots & \vdots & \vdots & \ddots
	\end{matrix}\right]
\end{equation}
The \(n\)-th row of the coefficient matrix is the coefficients for the \((n-1)\)-th finite difference.
The following result is a trivial consequence of row reduction:
\[T=CS\]
Substituting in
\[[T|CB]\]
as the augmented matrix, we can then use back-substitution to solve for the coefficients. For example, taking \(n=4\) samples yields
\[
	[CS\vert CB]=\left[\begin{matrix} 1 & 1 & 1 & 1\\
			0 & 1 & 3 & 7\\
			0 & 0 & 2 & 12\\
			0 & 0 & 0 & 6\\
		\end{matrix}
		\,\,\middle\vert\,\,\begin{matrix}
			B_1\\
			B_2-B_1\\
			B_3-B_2+B_1\\
			B_4-3B_3+3B_2-B_1\\
	\end{matrix}\right]
		\]
Thus,
\[6a_3 s^{2*3}= B_4-3B_3+3B_2-B_1\]
We can then substitute the result for \(a_3\) into the next equation
\[2a_2 s^{2*2}+12a_3s^{2*3} = B_3-B_2+B_1\]
and so forth until all of the coefficients are determined.

\subsubsection{Computation}
The initial Vandermond matrix can be calculated as
\[S_{ij}=j^{i-1}\]
The rows of the coefficient matrix can be calculated as follows:
\[C_{1j}=(-1)^j\]
\[C_{i+1,j}=-C_{ij}*\frac{j-i}{i+1}\]
This comes from the fact that
\[C_{ij}=(-1)^{i+j}\ncr{j}{i}\]
and
\[\ncr{n}{k+1}=\ncr{n}{k}*\frac{n-k}{k+1}\]

Finally, the background values \(B_i\) are calcualted as follows. For the \(i\)-th entry, the set of all points \(i*s^2\) away from the origin is generated using the midpoint ellipse algorithm (a way to rasterize an ellipse). The entry \(B_i\) is then taken to be the median brighness among all of these points, as to avoid the skewing effect of very bright flakes on the mean.

These matrices are then multiplied together, and back-substitution is used to solve for the coefficients. 

Once the coefficients are determined, the background fit can be determined recursively as:
\[f_0=a_N\]
\[f_{n} =s^2f_{n-1} +a_{N-n}\]
with
\[B_N(s)=f_N\]


\section{Morphological Transformations}
\section{Bounding Boxes}
\section{Determining Thickness}
\end{document}
