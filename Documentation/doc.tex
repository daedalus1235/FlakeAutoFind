\documentclass{article}
\usepackage{hw}

\title{FlakeAutoFind}
\author{Charles Yang}
\date{Last updates: \today}

\begin{document}
\maketitle

\setcounter{tocdepth}{1}
\tableofcontents

\section{Flattening}
The symmetry of the background is assumed to that of an ellipse. Similar to a circle, where the radius is defined
\[s^2=(x-x_c)^2+(y-y_c)^2\]
we define a distance coordinate
\[s^2=(x-x_c)^2+r^2(y-y_c)^2\]
where \(r\) is the aspect ratio of the ellipse, given
\[r = \frac{\text{width}}{\text{height}}\]

In this way, we consider the background to be a function of the elliptical distance. To maintain smoothness, the background is modeled as the sum of even terms:
\[B_N(s)=\sum_{n=0}^N a_ns^{2n}\]

Clearly, the first \(n\) coefficients can be solved for using a system of linear equations generated by \(n\) samples of the background. 

For simplicity, we consider only the lowest two orders, 
\[B_2(s) = a_0+a_1s^2\]
Taking two samples,
\[B_2(s_1) = a_0+a_1s_1^2\]
\[B_2(s_2) = a_0+a_1s_2^2\]
\[B_2(s_2)-B_2(s_1)=a_1(s_2^2-s_1^2)\]
\[a_1=\frac{B_2(s_2)-B_2(s_1)}{s_2^2-s_1^2}\]
\[a_0=B_2(s_1)-a_1s_1^2=B_2(s_2)-a_1s_2^2\]
\end{document}
